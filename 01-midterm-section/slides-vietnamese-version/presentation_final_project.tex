%!TeX encoding = UTF-8
%!TeX program = xelatex
\documentclass[notheorems, aspectratio=54]{beamer}
% aspectratio: 1610, 149, 54, 43(default), 32
\usepackage[utf8]{inputenc}
\usepackage[english, vietnamese]{babel}
\usepackage{latexsym}
\usepackage{amsmath,amssymb}
\usepackage{mathtools}
\usepackage{color,xcolor}
\usepackage{graphicx}
\usepackage{algorithm}
\usepackage{amsthm}
\usepackage{lmodern} % 解决 font warning
% \usepackage[UTF8]{ctex}
\usepackage{animate} % insert gif

\usepackage{lipsum} % To generate test text 
\usepackage{ulem} % 下划线,波浪线

\usepackage{listings} % display code on slides; don't forget [fragile] option after \begin{frame}

% ----------------------------------------------
% tikx
\usepackage{framed}
\usepackage{tikz}
\usepackage{pgf}
\usetikzlibrary{calc,trees,positioning,arrows,chains,shapes.geometric,%
	decorations.pathreplacing,decorations.pathmorphing,shapes,%
	matrix,shapes.symbols}
\pgfmathsetseed{1} % To have predictable results
% Define a background layer, in which the parchment shape is drawn
\pgfdeclarelayer{background}
\pgfsetlayers{background,main}

% define styles for the normal border and the torn border
\tikzset{
	normal border/.style={orange!30!black!10, decorate, 
		decoration={random steps, segment length=2.5cm, amplitude=.7mm}},
	torn border/.style={orange!30!black!5, decorate, 
		decoration={random steps, segment length=.5cm, amplitude=1.7mm}}}

% Macro to draw the shape behind the text, when it fits completly in the
% page
\def\parchmentframe#1{
	\tikz{
		\node[inner sep=2em] (A) {#1};  % Draw the text of the node
		\begin{pgfonlayer}{background}  % Draw the shape behind
			\fill[normal border] 
			(A.south east) -- (A.south west) -- 
			(A.north west) -- (A.north east) -- cycle;
\end{pgfonlayer}}}

% Macro to draw the shape, when the text will continue in next page
\def\parchmentframetop#1{
	\tikz{
		\node[inner sep=2em] (A) {#1};    % Draw the text of the node
		\begin{pgfonlayer}{background}    
			\fill[normal border]              % Draw the ``complete shape'' behind
			(A.south east) -- (A.south west) -- 
			(A.north west) -- (A.north east) -- cycle;
			\fill[torn border]                % Add the torn lower border
			($(A.south east)-(0,.2)$) -- ($(A.south west)-(0,.2)$) -- 
			($(A.south west)+(0,.2)$) -- ($(A.south east)+(0,.2)$) -- cycle;
\end{pgfonlayer}}}

% Macro to draw the shape, when the text continues from previous page
\def\parchmentframebottom#1{
	\tikz{
		\node[inner sep=2em] (A) {#1};   % Draw the text of the node
		\begin{pgfonlayer}{background}   
			\fill[normal border]             % Draw the ``complete shape'' behind
			(A.south east) -- (A.south west) -- 
			(A.north west) -- (A.north east) -- cycle;
			\fill[torn border]               % Add the torn upper border
			($(A.north east)-(0,.2)$) -- ($(A.north west)-(0,.2)$) -- 
			($(A.north west)+(0,.2)$) -- ($(A.north east)+(0,.2)$) -- cycle;
\end{pgfonlayer}}}

% Macro to draw the shape, when both the text continues from previous page
% and it will continue in next page
\def\parchmentframemiddle#1{
	\tikz{
		\node[inner sep=2em] (A) {#1};   % Draw the text of the node
		\begin{pgfonlayer}{background}   
			\fill[normal border]             % Draw the ``complete shape'' behind
			(A.south east) -- (A.south west) -- 
			(A.north west) -- (A.north east) -- cycle;
			\fill[torn border]               % Add the torn lower border
			($(A.south east)-(0,.2)$) -- ($(A.south west)-(0,.2)$) -- 
			($(A.south west)+(0,.2)$) -- ($(A.south east)+(0,.2)$) -- cycle;
			\fill[torn border]               % Add the torn upper border
			($(A.north east)-(0,.2)$) -- ($(A.north west)-(0,.2)$) -- 
			($(A.north west)+(0,.2)$) -- ($(A.north east)+(0,.2)$) -- cycle;
\end{pgfonlayer}}}

% Define the environment which puts the frame
% In this case, the environment also accepts an argument with an optional
% title (which defaults to ``Example'', which is typeset in a box overlaid
% on the top border
\newenvironment{parchment}[1][Example]{%
	\def\FrameCommand{\parchmentframe}%
	\def\FirstFrameCommand{\parchmentframetop}%
	\def\LastFrameCommand{\parchmentframebottom}%
	\def\MidFrameCommand{\parchmentframemiddle}%
	\vskip\baselineskip
	\MakeFramed {\FrameRestore}
	\noindent\tikz\node[inner sep=1ex, draw=black!20,fill=white, 
	anchor=west, overlay] at (0em, 2em) {\sffamily#1};\par}%
{\endMakeFramed}

% ----------------------------------------------

\mode<presentation>{
	\usetheme{CambridgeUS}
	% Boadilla CambridgeUS
	% default Antibes Berlin Copenhagen
	% Madrid Montpelier Ilmenau Malmoe
	% Berkeley Singapore Warsaw
	\usecolortheme{beaver}
	% beetle, beaver, orchid, whale, dolphin
	\useoutertheme{infolines}
	% infolines miniframes shadow sidebar smoothbars smoothtree split tree
	\useinnertheme{circles}
	% circles, rectanges, rounded, inmargin
}
% 设置 block 颜色
\setbeamercolor{block title}{bg=red!30,fg=white}

\newcommand{\reditem}[1]{\setbeamercolor{item}{fg=red}\item #1}

% 缩放公式大小
\newcommand*{\Scale}[2][4]{\scalebox{#1}{\ensuremath{#2}}}

% 解决 font warning
\renewcommand\textbullet{\ensuremath{\bullet}}

% ---------------------------------------------------------------------
% flow chart
\tikzset{
	>=stealth',
	punktchain/.style={
		rectangle, 
		rounded corners, 
		% fill=black!10,
		draw=white, very thick,
		text width=6em,
		minimum height=2em, 
		text centered, 
		on chain
	},
	largepunktchain/.style={
		rectangle,
		rounded corners,
		draw=white, very thick,
		text width=10em,
		minimum height=2em,
		on chain
	},
	line/.style={draw, thick, <-},
	element/.style={
		tape,
		top color=white,
		bottom color=blue!50!black!60!,
		minimum width=6em,
		draw=blue!40!black!90, very thick,
		text width=6em, 
		minimum height=2em, 
		text centered, 
		on chain
	},
	every join/.style={->, thick,shorten >=1pt},
	decoration={brace},
	tuborg/.style={decorate},
	tubnode/.style={midway, right=2pt},
	font={\fontsize{10pt}{12}\selectfont},
}
% ---------------------------------------------------------------------

% code setting
\lstset{
	language=C++,
	basicstyle=\ttfamily\footnotesize,
	keywordstyle=\color{red},
	breaklines=true,
	xleftmargin=2em,
	numbers=left,
	numberstyle=\color[RGB]{222,155,81},
	frame=leftline,
	tabsize=4,
	breakatwhitespace=false,
	showspaces=false,               
	showstringspaces=false,
	showtabs=false,
	morekeywords={Str, Num, List},
}


%
\author{Vương Gia Bảo, Nguyễn Viết Dũng\newline Ngô Xuân Kiên, Lê Nhựt Nam}
\title{NHẬP MÔN MÁY HỌC \newline SPEAKER RECOGNITION FROM \newline RAW WAVEFORM WITH SINCNET}
\institute{Đại học Khoa học Tự nhiên, Đại học Quốc gia TP HCM} 


%
\setbeamertemplate{caption}[numbered]
% \setbeamertemplate{footline}[frame number]
% footer
\makeatletter
\setbeamertemplate{footline}
{
	\leavevmode%
	\hbox{%
		\begin{beamercolorbox}[wd=1\paperwidth,ht=2.25ex,dp=1ex,right]{institute in head/foot}%
			\usebeamerfont{title in head/foot} 
			\insertframenumber{} / \inserttotalframenumber\hspace*{2ex} 
	\end{beamercolorbox}}%
}
\makeatother
%
\newcommand{\argmax}{\arg\!\max}

\begin{document}

\begin{frame}
\titlepage
\end{frame}

\begin{frame}{Nội dung trình bày}
\tableofcontents
\end{frame}


\section{Động lực nghiên cứu khoa học}
\begin{frame}{Động lực nghiên cứu khoa học}
	\begin{itemize}
		\item Về lĩnh vực Nhận dạng giọng nói, đây là một lĩnh vực nghiên cứu có nhiều ứng dụng vào thực tiễn đời sống như xác thực sinh trắc học, pháp y, bảo mật, nhận dạng giọng nói và phân cực giọng nói.
		\item Các phương pháp SOTA dựa trên việc biểu diễn i-vectors của những đoạn
		giọng nói, cải thiện đáng kể so với mô hình Gaussian Mixture Model-Universal Background Models
		\item Sự phát triển của Deep Neural Networks
	\end{itemize}
\end{frame}

\section{Phát biểu bài toán}
\begin{frame}{Phát biểu bài toán}
	Hai tác vụ lớn trong lĩnh vực Nhận dạng Giọng nói:\newline
	\begin{itemize}
		\item Tác vụ: Định danh người nói (Speaker Identification)
		\begin{itemize}
			\item Đầu vào (Input): Dữ liệu âm thanh giọng nói
			\item Đầu ra (Output): Danh tính của người nói
		\end{itemize}
		\item 	Tác vụ: Xác minh người nói (Speaker Verification)
		\begin{itemize}
			\item Đầu vào (Input): Dữ liệu âm thanh giọng nói
			\item Đầu ra (Output): Đồng ý/ Từ chối
		\end{itemize}
	\end{itemize}
\end{frame}
\section{Giới thiệu về bài báo}
\begin{frame}{Giới thiệu về bài báo}
	\begin{itemize}
		\item Bài báo đề xuất một kiến trúc mạng CNN (Convolutional Neural Networks) mới, khai phá lớp tích chập đầu tiên để khám phá nhiều thông tin hơn - SincNet.
		\item SincNet dựa trên các hàm sinc được tham số hóa, để cài đặt các bộ lọc băng thông.
		\item Ngược lại với CNNs chuẩn, học tất cả các phần tử của mỗi bộ lọc filter, ở đây, SincNet chỉ dùng các tần số cắt thấp và cao học trực tiếp dữ liệu với phương pháp đề xuất.
		\item Nhỏ gọn và hiệu quả trong việc tùy chỉnh với ứng dụng mà chúng ta muốn.
		\item Sự kết hợp tuyệt vời giữa Học máy (Machine Learning) và Xử lý Tín hiệu số (Digital
		Signal Processing).
	\end{itemize}
\end{frame}

\section{Vấn đề khi xử lý tín hiệu giọng nói}
\begin{frame}{Vấn đề khi xử lý tín hiệu giọng nói}
	\begin{itemize}
		\item Dữ liệu với số chiều cao
		\begin{figure}[H]
			\includegraphics[width=1\linewidth]{images/capture_01.png}
			\caption{A brief introduction to SincNet - Mirco Ravanelli}
			\label{fig:writing-thesis}
		\end{figure}
	\end{itemize}
\end{frame}
\begin{frame}{Vấn đề khi xử lý tín hiệu giọng nói}
	\begin{itemize}
		\item Những đặc trưng giọng nói có thể bị mất đi trong quá trình xử lý sóng thô.
		\begin{figure}[H]
			\includegraphics[width=0.9\linewidth]{images/perceptual_evidence.png}
			\caption{A brief introduction to SincNet - Mirco Ravanelli}
			\label{fig:writing-thesis}
		\end{figure}
	\end{itemize}
\end{frame}
\begin{frame}{Vấn đề khi xử lý tín hiệu giọng nói}
	\begin{itemize}
		\item Sự biến mất độ dốc đạo hàm - vanishing gradient
		\begin{figure}[H]
			\includegraphics[width=0.75\linewidth]{images/Vanishing-Gradients-in-DNN.png}
			\caption{Vanishing Gradient problem in Deep Neural Networks}
			\label{fig:writing-thesis}
		\end{figure}
	\end{itemize}
\end{frame}
\begin{frame}{Vấn đề khi xử lý tín hiệu giọng nói}
	\begin{itemize}
		\item Các bộ lọc CNNs thường có những hình dạng đa băng tần không hợp lý, khó hiểu với con người.
		\begin{figure}[H]
			\includegraphics[width=0.9\linewidth]{images/interpretability_problems.png}
			\caption{A brief introduction to SincNet - Mirco Ravanelli}
			\label{fig:writing-thesis}
		\end{figure}
	\end{itemize}
\end{frame}

\section{Kiến trúc mạng SincNet}
\begin{frame}{SincNet}
	\begin{block}{~\vspace{0.7cm}}
		\begin{center}
			\vspace{-0.8cm}
			\begin{tabular}{p{0.45\textwidth}|p{0.45\textwidth}}
				\textcolor{white}{\bf Standard CNN} & \textcolor{white}{\bf SincNet} \\\\
				 $y[n] = x[n] * h[n]$ & $y[n] = x[n] * g[n, \theta]$\\
				 Học từ tất cả các phần tử của mỗi filter & Chỉ học từ tham số $\theta$ của một kernel được định nghĩa từ trước\\
			\end{tabular}
		\end{center}
	\end{block}
	\textbf{Vấn đề đặt ra: Làm thế nào để chọn được hàm $g(.)$?}
\end{frame}
\begin{frame}{SincNet}
	Ý tưởng để xây dựng hàm $g(.)$ đến từ Xử lý tín hiệu số, dựa trên bộ lọc băng thông (\textbf{band-pass filters}), chỉ có tần số cắt thấp và cao (low-high cutoff frequencies) được học
	\begin{figure}[H]
		\includegraphics[width=0.9\linewidth]{images/band_passfilters.png}
		\caption{A brief introduction to SincNet - Mirco Ravanelli}
		\label{fig:writing-thesis}
	\end{figure}
\end{frame}
\begin{frame}{SincNet}
	Với $f_1$ $f_2$ lần lượt là tần số cắt thấp (low) và cao (high) đã được học, $rect(.)$ là hàm rectangular trong miền tần số.
	
	$$G[f, f_1, f_2] = rect\left(\frac{f}{2f_2}\right) -  rect\left(\frac{f}{2f_1}\right)$$
	
	Để có thể trở lại miền thời gian được, ta sử dụng phép biến đổi Fourier Ngược. Thu được
	
	$$g[n, f_1, f_2] = 2f_2sinc(2\pi f_2 n) - 2f_1sinc(2\pi f_1 n)$$ 
\end{frame}
\begin{frame}{SincNet}
	\begin{block}{~\vspace{0.7cm}}
		\begin{center}
			\vspace{-0.8cm}
			\begin{tabular}{p{0.45\textwidth}|p{0.45\textwidth}}
				\textcolor{white}{\bf CNN Filters} & \textcolor{white}{\bf SincNet Filters} \\\\
				\includegraphics[width=1\linewidth]{images/cnn_filters.png} & \includegraphics[width=1\linewidth]{images/sincnet_filters.png}\\
			\end{tabular}
		\end{center}
	\end{block}
\end{frame}
\begin{frame}{Kiến trúc mạng SincNet}
	\begin{figure}[H]
		\includegraphics[width=0.4\linewidth]{images/SincNet.png}
		\caption{The SincNet Architecture}
		\label{fig:writing-thesis}
	\end{figure}
\end{frame}
\begin{frame}{Đặc điểm mô hình mạng SincNet}
	\begin{columns}
		\begin{column}{0.47\textwidth}
			\begin{figure}[H]
				\includegraphics[width=0.9\linewidth]{images/SincNet.png}
			\end{figure}
		\end{column}
		\begin{column}{0.5\textwidth}
		\begin{itemize}
			\item Tính hội tụ nhanh: Khả năng hội tụ nhanh hơn CNN, ngay từ những epochs đầu tiên.
			\begin{figure}[H]
				\includegraphics[width=0.9\linewidth]{images/fast_convergence.png}
				\caption{Learning curves}
			\end{figure}
		\end{itemize}
		\end{column}
	\end{columns}
\end{frame}
\begin{frame}{Đặc điểm mô hình mạng SincNet}
	\begin{columns}
		\begin{column}{0.47\textwidth}
			\begin{figure}[H]
				\includegraphics[width=0.9\linewidth]{images/SincNet.png}
			\end{figure}
		\end{column}
		\begin{column}{0.5\textwidth}
			Tính hiệu quả
			\begin{figure}[H]
				\includegraphics[width=0.35\linewidth]{images/g_symmetric.png}
			\end{figure}
			Các hàm kernel $g(.)$ là đối xứng nên chỉ cần thực hiện phép tích chập trên một phần filter $\Leftrightarrow$ Điều này sẽ tiết kiệm 50\% việc tính toán.
		\end{column}
	\end{columns}
\end{frame}
\begin{frame}{Đặc điểm mô hình mạng SincNet}
	\begin{columns}
		\begin{column}{0.47\textwidth}
			\begin{figure}[H]
				\includegraphics[width=0.9\linewidth]{images/SincNet.png}
			\end{figure}
		\end{column}
		\begin{column}{0.5\textwidth}
			Cần ít tham số cho việc huấn luyện mô hình. \newline
			Giả sử ta gọi: $F$ là số bộ lọc filters, $L$ kích thước của mỗi bộ lọc  filters
			\begin{itemize}
				\item Với CNN, số tham số là $parameters = F * L$
				\item Với SincNet, số tham số là $parameters = 2F$
			\end{itemize}
		\end{column}
	\end{columns}
\end{frame}
\begin{frame}{Đặc điểm mô hình mạng SincNet}
	\begin{columns}
		\begin{column}{0.47\textwidth}
			\begin{figure}[H]
				\includegraphics[width=0.9\linewidth]{images/SincNet.png}
			\end{figure}
		\end{column}
		\begin{column}{0.5\textwidth}
			\begin{itemize}
				\item Tính diễn giải: Biểu đồ đáp ứng xung tích lũy của SincNet cho thấy nhiều đặc trưng quan trọng của giọng nói. 
				\begin{figure}[H]
					\includegraphics[width=0.9\linewidth]{images/interpretability.png}
				\end{figure}
			\end{itemize}
		\end{column}
	\end{columns}
\end{frame}

\section{Các kết quả}
\begin{frame}{Các kết quả đối với tác vụ Speaker Identification}
\begin{figure}[H]
	\centering
	\includegraphics[width=0.75\textwidth]{images/performance_speaker_identification.png}
	\caption{Bảng kết quả SicNet trong tác vụ nhận dạng giọng nói - SI}
	\label{fig:writing-thesis}
\end{figure}	
\end{frame}
\begin{frame}{Các kết quả đối với tác vụ Speaker Verification}
\begin{figure}[H]
	\centering
	\includegraphics[width=0.75\textwidth]{images/performance_speaker_verification.png}
	\caption{Bảng kết quả SicNet trong tác vụ xác minh giọng nói - SV}
	\label{fig:writing-thesis}
\end{figure}	
\end{frame}
\begin{frame}{Các kết quả đối với tác vụ SV i-vectors}
	Hệ thống \textbf{i-vector} tốt nhất của nhóm tác giả đạt được \textbf{EER = 1,1\%}, \textbf{khá xa so với những gì đạt được với hệ thống DNN}.	
\end{frame}
\begin{frame}{Kết quả Speaker Recognition trên DIRHA}
	\begin{figure}[H]
		\centering
		\includegraphics[width=0.65\textwidth]{images/sr_sincnet_result.png}
		\caption{Bảng kết quả Speaker Recognition SincNet trên tập DIRHA}
		\label{fig:writing-thesis}
	\end{figure}
	SincNet hoạt động rất tốt trên DIRHA, so với các kỹ thuật khác, nó luôn có độ lỗi thấp nhất.	
\end{frame}

\section{Kết luận}
\begin{frame}{Kết luận}
	Về SincNet Architecture
	\begin{itemize}
		\item Cơ sở lý thuyết Toán học vững vàng: Kỹ thuật band-pass filter, Window trong Xử lý tín hiệu số.
		\item Tính toán nhanh và gọn nhẹ: Như đã nói, đây là một đặc điểm của SincNet nhờ vào dùng ít tham số, kernel đối xứng.
		\item Kết hợp với Deep Learning một cách hiệu quả.
		\item Sử dùng DNN-Class trong đánh giá, cho kết quả đầy hứa hẹn, có độ lỗi EER thấp
	\end{itemize}
	Nhưng vẫn có hạn chế
	\begin{itemize}
		\item DNN-class tuy có EER thấp nhưng đánh đổi nhiều sự linh hoạt so với d-vectors
	\end{itemize}
\end{frame}
\begin{frame}{Kế hoạch dự kiến cuối kỳ}
	\begin{itemize}
		\item Chuẩn bị dữ liệu (cho tiếng Anh và \textbf{tiếng Việt}).
		\item Xây dựng mô hình.
		\item Trực quan hóa, đánh giá mô hình.
		\item Nhận xét những kết quả làm được.
	\end{itemize}
\end{frame}

\section{Tài liệu tham khảo}
\begin{frame}{Tài liệu tham khảo}
	\nocite{*}
	\bibliography{references}\newpage\cleardoublepage
	\bibliographystyle{plain}
\end{frame}


\begin{frame}{Q\&A}
	\begin{center}
		\Huge Cảm ơn thầy và các bạn đã theo dõi và lắng nghe
	\end{center}
\end{frame}

\end{document}